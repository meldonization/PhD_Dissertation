\begin{acknowledgement}

感谢我的导师 --- 南京大学周济林教授。「世有伯樂,然後有千里馬。千里馬常有,而伯樂不常有」。
周老师处处为人师表,万事皆与人为善,平日视学生如己出,为课题组包括我提供了诸多机遇和资源。
尤记大三下学期的天体力学基础课,周老师风采奕奕、旁征博引,令台下皆无不酣畅淋漓,有幸我
能同与之。感激能得周老师的信任与栽培,我才能开启系外行星研究这扇大门。平日,他坚定不移地
让我遨游于宽松自在的科研环境中,又同时对科研无处不严谨细致。所谓「以銅為鏡,可以正衣冠;
以史為鏡,可以知興衰;以人為鏡,可以知得失」。周老师的指导与恩泽,我定将永铭于心,秉承光扬 。

\vspace{1em}

感谢我的联合导师 Douglas Lin,Alice Quillen 与周旭老师,她们的细心指导让我明白了科研并非一蹴
而就,多注重积累与思维方能有为。她们更在我举目无亲之时对我生活关爱照顾,感激涕零,将爱予之。
在读博之初,正因有谢基伟副教授的入门指导,我才能飞速地学到新知识,充实了本硕交界的关键时期。
同样,如若没有同组孙义遂院士,张辉、刘慧根副教授的指导;陈媛媛、杨佳祎、王松虎、王颖等师姐
师兄的讨论,杨明、王楠等同学的照应与余周毅、梁恩思、吴东红等学妹学弟的陪伴,我应当难以完成
这段研究生时光。同样,也感谢学院所有科研人员与行政人员悉心对待我读博期间发生的大小琐事。

\vspace{1em}

此外,本文作者感谢其父母在计划生育政策下毅然绝然地将我诞生到人世间,才不至错失人间的种种。
谢谢我的两位姐姐以及我的伴侣,我爱你们赤忱如星。与此同时,也要感谢所有的亲朋好友,郑重地对你
们一如既往的支持说声谢谢。最后,感谢此论文模板 NJU-Thesis 的开发者胡海星。

\afterpage{


上文未出现名字的感谢人员姓名清单(与本人接触先后排序)如下:
宗玉凤、
孟彩安、
孟红梅、
孟红珠、
童希、
徐斌、
陆鹏、
李彦哲、
张迪衡、
梁钊宁、
张宁潇、
刘逸清、
白云高、
高宇翔、
呼延宗泊、
胡一鸣、
向俊夫、
王凯、
Damien Hua、
杨艳、
郝奇、
侯锡云、
吴伟、
宋晨晖、
郭艳、
王鑫、
周鑫、
王平、
赵刚、
王素、
宫衍香、
李道海、
Rob Wittenmyer、
王子逸、
欧建文、
Serafini John、
伍业刚、
Areeya Chantasri、
王崇旭、
Cassius Ellis、
Alice Chi、
Laura Lorber、
张晓佳、
张曦、
王卓骁、
Florence Koh、
Thomas Trio、
Lesley Harrison、
Alex Santana、
Mark Vee、
宋禹、
吴沛刚、
苏湘宁、
陈迪昌、
朱佳鹏、
王奇等。感谢她们在我博士期间给予的关心与照顾。
}

\end{acknowledgement}
