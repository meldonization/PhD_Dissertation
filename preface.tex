%% Zeyang Meng PhD Thesis 2017.
%%
\begin{preface} 

\epigraph{... Der bestirnte Himmel über mir, und das moralische Gesetz in mir.}{\textit{Immanuel Kants}}

\vspace{1cm}

自诞生以来,人类就从未停止向外界探索以及反思自我。从技术层面上看,在望远镜
分辨率与灵敏度被推向极限的同时,科学家们也在考究着生命与宜居行星存在的合理
性。现如今太阳系外行星(简称:系外行星)科学领域,与传统的行星科学相比,虽
然诞生并未满半个世纪,但也正尝试以「它山之石」 -- 其它恒星周围的行星们,来内
窥太阳系行星系统以及生命本身。
\\ \par
近二十年来,随着观测技术的迭代更新,被探测到的系外行星样本也在正加速地扩大,
以往得到的基于太阳系行星系统的传统行星形成理论也随之不停地被改进、更正。正
当行星的形成阶段一次次地被观测到(如行星形成早期的原行星盘),一些看似与太
阳系大不同的系外行星系统(如周期小于十天的热木星)也不断地完善并细化着已知
的行星系统动力学作用理论。系外行星领域也正随之如此恰似涓涓细流般启发着人类
对太阳系起源与系外生命的认知。
\\ \par
而以上这一切里程碑,正离不开高精度数据预处理与前卫的理论分析和计算,如基于
地面的高精度光谱仪 HIRES 和 $Kepler$ 太空望远镜等仪器的数据处理流程。此册博
士论文也借此尝试从高精度数据处理入手(「中国之星」 中的「鬼像」处理),去探
寻可能的系外原始行星盘(搜索大麦哲伦云星系中的掩食盘候选体),并且从统计上
对现有的热木星系统的形成与潮汐演化作出部分限制(平衡潮汐模型下,热木星轨道
法向与宿主恒星自转轴取向不一致性在统计上的物理性质)。
\\ \par
作为一门跨学科领域,如今系外行星也正通往着融合了天体生物和行星大气等多科学
的方向发展。Spitzer 太空望远镜也已观测到数十颗系外行星大气存在的证据。在不久
的将来,光合化学与有机分子化学也会被有效地应用到如何将观测到的光信号转化成
生命存在的证据。科学家们也正在摸索中走出一条通往太阳系的起源与太阳系外生命
之路,而此刻的人类是否已蓄势待发准备好迎接新的纪元...


\vfill

\hfill 孟泽洋

\hfill 丁酉年正月廿七

\hfill 南京大学

\end{preface}
