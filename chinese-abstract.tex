% 设置论文的中文摘要
% 设置中文摘要页面的论文标题及副标题的第一行。
% 此属性可选,其默认值为使用|\title|命令所设置的论文标题
%% \abstracttitlea{}
% 设置中文摘要页面的论文标题及副标题的第二行。
% 此属性可选,其默认值为空白
%% \abstracttitleb{}

% 论文的中文摘要
\begin{abstract}

天文学是一门建立在观测基础之上的研究,系外行星领域也同样不例外。探测与刻画系外行
星必须依赖精确稳定的光子捕捉系统、广域的空间覆盖度以及更长的时间基线。从第一颗主
序星周围的行星 51 Peg b 被人类捕捉探测,至如今总共约 3500 颗已知的系外行星,强大的
探测仪器和精确的数据处理已经逐步将行星科学从太阳系推向了太阳临近的星际空间,与此
同时一大批全新的科学疑问也接踵而至。

在本文的第一章节,首先引入南极天文相关的观测特点,接着我们从「中国之星」小型望远
镜阵列出发,通过分析该望远镜于 2008 年极夜在南极 Dome A 站点观测得到的总数超过三十
万张的 $i$ 波段测光图片数据,我们发现其中的「鬼像」效应会在很大程度上影响恒星的测光
精度。所以我们从该望远镜的设计光路图出发,摸索出「鬼像」产生的原因后。从源头上,
将产生「鬼像」的亮源意一一找出,并且对比得到两者的强度关系。并且从每一张原始数
据出发,定量求得被「鬼像」影响的背景恒星流量的变化和两者之间距离的经验关系。最后
回归到星表数据中修复「鬼像」对测光精度带来的影响。经过包括「鬼像」以及其它处理修正
后释放的数据成功被用在时域天文上双星、系外行星以及恒星耀斑等多种搜索研究上。诸如此
类的后续数据分析也更加有利于使用 AST3-2 南极天文望远镜定制系外行星搜索计划。

随后,我们便在第二章节中展开了对光学引力透镜实验三(OGLE-III)望远镜数据内掩食盘的
时域光变相关搜索。探测星周盘对于研究恒星与行星的早期形成过程具有非常重要的意义。利
用 OGLE-III 中大、小麦哲伦云以及银盘观测得的掩食双星(EBs)数据,我们从大麦云中共挑
选出 2823 个高测光精度、不相接深掩食的子样本。在分析过主、次掩食内星等分布统计二极
矩后,我们发现峰度(kurtosis)和斜度(skewness)可被用来区分掩食盘候选体与普通 EB 
系统。在判据 $|S| < 0.5$ 和 $K > 0$ 下,其中两个掩食盘候选体 OGLE-LMC-ECL-11893 与 
OGLE-LMC-ECL-17782 被程序成功识别。与此同时,在小麦云和银盘样本中并未发现同类掩
食盘候选。在所有 I 波段星等小于 19 等的大麦云 EB 样本中,我们估算得出此类不寻常的掩食
盘发生的概率大约为 1/1000,而且此概率随着样本恒星年龄的增长会显著减小。另外,在对
OGLE-LMC-ECL-17782 详细拟合分析后,我们发现由于此系统周期为 13.3 天,其内的掩食
盘距离主星(B 型星)仅有约 0.1 AU,主星的高光度将会导致星周盘温度(6000 K)超过内
部固体物质的升华点,因此急迫需要额外的观测与物理解释。

此外,本文于第四章节内汇总统计了现如今所有已被发现的系外行星系统。对于其中的热木星
系统(HJs),以往的研究显示它们的起源对于行星形成的动力学演化过程十分重要。然而观测
上显示在有效温度较高的中等质量恒星(HS)与温度较低的类太阳恒星(CS)周围这些系统
有着截然不同的性质。Rossiter-McLaughlin 效应的测量统计显示 HS 周围 HJ 的恒星自转与行
星轨道法向角度($\Psi$)为随机分布在 0 至 180$^\circ$ 之间,而 CS 系统取向则一致($\Psi
\simeq 0$)。以往的解释认为以上现象是由于这两类恒星的对流外包层厚度不同,从而主星的
弱摩擦近似下的平衡潮汐耗散因子$Q$ 值有所差异而导致的。比如 CS 潮汐耗散率较高且自转
速度偏慢,所以主星自转轴的取向更容易通过行星的潮汐作用而被重新扭转。本文重新回顾了
这些机制,并且同时考虑了恒星星风与平衡潮汐作用这两个效应。我们发现对于 HS,观测得
到的 $\Psi$ 角并没有与恒星的自转呈现相关性,且 CS 中行星的轨道位置也并不足以能够改
变主星的自转角动量。这些发现与以往的理论解释十分不符,因而现有的单一的 $Q$ 值来源
理论不能解释所有的 HJ 系统。另外,我们的结果可将这些恒星的潮汐参数 $Q$ 值限定在约 
10$^6$ 或更大。

结尾部分,本文总结了如今系外行星的部分观测趋势,也提出一小部分理论上需要去完善的问
题,包括如何在全局上更完整地了解行星系统的演化历史。最后,讨论了迈入更智能切高精度
的数据处理方案。在不久的将来,对单个系外行星的认知一定将会更加科学完善,刻画亦会愈
发细致,「另一个地球」的迷雾已被逐步驱散,生命起源以及系外生命新发现与许就近在咫尺,
等着人类用崭新的思路去开拓与挖掘。


% 中文关键词。关键词之间用中文全角分号隔开,末尾无标点符号。
\keywords{方法:数据分析;双星:掩食;列表;行星与卫星:动力学演化和稳定性;行星---恒星相互作用}
\end{abstract}


