%% Appendix tex file by Meldonization.

\chapter{文内常用约定} \label{apdx:nomenclature}
\section{物理符号含义} \label{apdx:symbol}
\begin{multicols}{2}
\begin{tabularx}{0.85\linewidth}{@{\extracolsep{\fill}}lr}
\centering
$a$      	     			&     轨道半长径 		\\
$d$      	     			&     质点二体间距离	 	\\
$P$      	     			&     轨道周期	 		\\
$e$      	     			&     轨道偏心率 		\\
$i$          	     			&     轨道倾角 			\\
%$\omega$      			&     近心点角距 		\\
%$\Omega$     			&     升交点经度 		\\
%$f$              			&     真近点角   			\\
$\tif{M}_\odot$          		&     太阳质量   			\\
$\tif{R}_\odot$          		&     太阳半径   			\\
$\tif{M}_\tif{J}$          		&     木星质量   			\\
$\tif{M}_\oplus$          	&     地球质量   			\\
$M_\tif{s}$          		&     恒星质量   			\\
$R_\tif{s}$          		&     恒星半径   			\\
$M_\tif{p}$         	 	&     行星质量   			\\
$m$         	 			&     视星等   			\\
$c$         	 			&     真空光速   			\\
$\nu$         	 		&     光子频率   			\\
$k$         	 			&     玻尔兹曼常数   		\\
$h$         	 			&     普朗克常数   		\\
T		       	 		&      温度   			\\
$\tif{T}_\tif{p}$         	 	&      行星温度   		\\
$\tif{T}_\tif{s}$         	 	&      恒星温度   		\\
$\tif{T}_\tif{d}$         	 	&      星周盘温度   		\\
$\tif{T}_\tif{eff}$         	 	&      有效温度   		\\
L		         	 	&      光度		   		\\
$\tif{L}_\tif{IR}$         	 	&      红外光度   		\\

\end{tabularx}
\columnbreak

\begin{tabularx}{0.85\linewidth}{@{\extracolsep{\fill}}lr}
\centering

Z		       	 		&      金属丰度   		\\
U, B, V, I		       	 	&      Johnson 波段   		\\
$u,i$		       	 		&      SDSS 波段 		\\
$A_V$		       	 	&      V 波段消光系数   	\\
$\mu$ 				&	折合质量			\\


\end{tabularx}
\end{multicols}

\newpage


\section{首字母缩写}  \label{apdx:acronym}
本文首字母缩写主要参考自书籍\citen{AstroDict},按照字母先后顺序排列如下:
\begin{multicols}{2}
\begin{tabularx}{0.85\linewidth}{@{\extracolsep{\fill}}lr}
\centering
ALMA           &   阿卡塔马大型毫米波阵     	  \\ 
AST3           &   南极巡天望远镜     	   	  \\  
AU		   &	天文单位				  \\
BB		   &   黑体					   \\
CS 	 	   &   环恒星				   \\
CCD		   &   电荷耦合器件			   \\
COM		   &   质心					   \\
CSTAR        &   南极之星望远镜阵列 		   \\  
EB               &   掩食双星 				   \\ 
ED               &   不接食双星 		 	   \\ 
EDV             &   掩食深度变化             	   \\
ETV             &   掩食计时变化             	   \\
FFP             &   自由飘游行星        		   \\ 
FOV            &   视场			     		   \\ 
GB              &   银核球				   \\
GD              &   银盘				  	    \\
GI                &   引力不稳定              		    \\
HJ               &    热类木星              		    \\
HJD            &    日心儒略日 	           	    \\
HST            &    哈勃空间望远镜            	    \\
IAU             &    国际天文学联合会   	    	    \\
IR               &   近红外波段		 	            \\
LMC            &   大麦哲伦云		 	    \\
MMR           &   平运动共振   	                     \\   
MMSN         &   最小质量原行星盘                 \\
NIAOT         &   南京天文光学技术研究所       \\
NIR              &   近红外波段		             \\

\end{tabularx}
\columnbreak

\begin{tabularx}{0.85\linewidth}{@{\extracolsep{\fill}}lr}
\centering
OGLE         &    光学引力透镜实验   	     	     \\
PPD            &    原行星盘   	   	     	     \\
RV              &    视向速度                    	     \\
SDSS         &    斯隆巡天                   	     	     \\
SED            &    光谱能量分布                   	     \\
SMC            &   小麦哲伦云		 	    \\
TDV            &     凌星深度变化             	     \\
TNO            &    海外天体                    	     \\
TTV             &    凌星计时变化             	     \\
UT               &    世界时		             	     \\
YSO            &    初期恒星体                	     \\


\end{tabularx}
\end{multicols}




\chapter{二体运动} \label{apdx:twobodyproblem}

\begin{figure}[h]
\centering
\includegraphics[width=0.95\textwidth]{figures/appendix/f1_ellipse.pdf}
\caption{二体在轨道平面内的椭圆运动示意图,图片来源 Perryman。}
\label{fig:ellipse}
\end{figure}

在次方反比的中心引力作用下,二体的运动轨迹为封闭的圆锥曲线\cite{Newton1687}。
图\ref{fig:ellipse} 和 \ref{fig:3dorbit} 展示的是椭圆二体运动示意图,其中 $a,\,e,\,i,\,\omega,\,\Omega,\,f$ 
被称作描述椭圆二体运动的六个轨道常数,文内符号几乎均沿袭自书本\citen{MurrayDermott1999ssd}。
在中心天体坐标系中,行星距离主星的标量距离 $r$ 可表示为:
\begin{equation} \label{radialdistance}
r = \frac{a(1-e^2)}{1+e\cos f}
\end{equation} %\myequation{二体运动的径向距离与轨道根素的关系}

\begin{figure}[t]
\centering
\includegraphics[width=0.85\textwidth]{figures/appendix/f2_3dorbit.pdf}
\caption{椭圆二体运动轨道在三维空间内的示意图,图内标识分别为轨道根数与参考系。图片来源 Perryman。}
\label{fig:3dorbit}
\end{figure}

\begin{figure}[hb!]
\centering
\includegraphics[width=0.75\textwidth]{figures/appendix/f3_rochelimit.pdf}
\caption[双星中洛希瓣的示意图,伴星主星质量比 $\mu_2/\mu_1 = 0.2$,其中 $L_1$ 被称为第一朗格朗日点,穿过它的等引力势面(用数字 1 来标注)分别被称作两颗星的洛希半径。图片版权 C. D. Murray,S. F. Dermott]{双星中洛希瓣的示意图,伴星主星质量比 $\mu_2/\mu_1 = 0.2$,其中 $L_1$ 被称为第一朗格朗日点,穿过它的等引力势面(用数字 1 来标注)分别被称作两颗星的洛希半径。图片来源书籍 \citen{MurrayDermott1999ssd}。}
\label{fig:rocher}
\end{figure}

\chapter{赫罗图} \label{apdx:HRdiagram}
\begin{figure}[ht!]
\centering
\includegraphics[width=0.97\textwidth]{figures/appendix/f4_HRdiagram.pdf}
\caption[赫罗图示意图,数据采自 ESA/Hipparcos/Tycho]{太阳附近恒星赫罗图示意图,纵坐标 V 波段绝对星等,横坐标 B-V 色指数。另外按照色指数不同分别标注了恒星光谱型(BAFGKM --- 0),以及绝对星等对应的恒星光度(右侧纵坐标),可以看到临近恒星主要分为主序以及巨星支。 数据来源文献 \citen{Hipparcos1997}。}
\label{fig:hrdiagram}
\end{figure}


