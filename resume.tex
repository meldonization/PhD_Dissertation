\begin{resume}

\begin{cv}{}
\begin{cvlist}{个人信息}
\setlength{\itemsep}{-0.250\baselineskip plus 0.2em minus 0.2em}
\item[姓名] 孟泽洋 
\item[性别] 男
\item[出生年月] 1989 年 8 月 25 日
\item[籍贯]  \href{https://goo.gl/maps/rJ1jvfKt65z}{江苏省东台市}
\item[通讯地址]  江苏省南京市栖霞区仙林大道 163 号 \\
 			 南京大学仙林校区天文与空间科学学院,210089 
\item[联系方式]   (+86)~15950459632
\item[邮箱] \href{mailto:mengzy1989@gmail.com}{mengzy1989@gmail.com}
\item[个人主页]  \url{https://meldonization.github.io}
\end{cvlist}
\begin{cvlist}{教育背景}
\setlength{\itemsep}{-0.250\baselineskip plus 0.2em minus 0.2em}
\item[09/1995 - 06/2001] 江苏省东台市五烈镇木港小学
\item[09/2001 - 06/2004] 江苏省东台市五烈镇同安初级中学
\item[09/2004 - 06/2007] 江苏省东台市高级中学
\item[09/2007 - 06/2011] 南京大学天文系,获理学学士学位,江苏省南京市
\item[09/2013 - 01/2014] University of Rochester, NY, USA
\item[10/2015 - 10/2016] University of California, Santa Cruz, CA, USA
\item[09/2011 - 现在]  南京大学天文与空间科学学院天体力学与天体测量专业
\end{cvlist}
\begin{cvlist}{获奖情况}
\setlength{\itemsep}{-0.250\baselineskip plus 0.2em minus 0.2em}
\item[2010]  优秀学生
\item[2010]  国家励志奖学金
\item[2012]  研究生优秀奖学金
\item[2014]  研究生优秀奖学金
\end{cvlist}
\begin{cvlist}{学术会议}
\setlength{\itemsep}{-0.250\baselineskip plus 0.2em minus 0.2em}
\item[2011]   引力透镜暑期班暨研讨会
\item[2011]   中国天文学会 2011 年学术年会(报告)
\item[2012]   国际天文学联合会 2012 年大会(海报)
\item[2013]   中国天文学会 2013 年学术年会(报告)
\item[2015]   基本天文学的现状与挑战学术研讨会
\item[2016]   美国天文学年会 AAS meeting 228
\item[2016]   中国天文学会 2016 年学术年会(报告)
\end{cvlist}   
\begin{cvlist}{实测经历}
\setlength{\itemsep}{-0.250\baselineskip plus 0.2em minus 0.2em}
\item[2009]    盱眙观测站实测天体物理 V 波段测光数据处理
\item[2010]    紫金山天文台太阳塔实测以及 H$\alpha$ 吸收线光谱数据分析
\item[2011]    国家天文台暑期实习光谱数据时间序列分析
\item[2015] \href{http://aag.bao.ac.cn/klaws/index.php}{南极 Dome A 观测站 AST3-1 望远镜极夜远程观测 20+ 夜晚}
\item[2016]  \href{http://aag.bao.ac.cn/klaws/index.php}{南极 Dome A 观测站 AST3-2 望远镜极夜远程观测 60+ 夜晚}
\end{cvlist}   
\begin{cvlist}{专业技能}
\item Python(熟练) $\bullet$ 中英双语(中文母语以及六级英语) $\bullet$ C 语言(计算机二级) $\bullet$ Fortran(数值计算) $\bullet$ Linux(精通) $\bullet$ 数据收集与分析(信号处理、远程操作与时间序列分析)
\end{cvlist}   
\begin{cvlist}{其他}
\setlength{\itemsep}{-0.250\baselineskip plus 0.2em minus 0.2em}
\item[2011]  南京大学青年志愿者协会渊声巷小学主题活动
\item[2015]  \href{http://www.ifanr.com/author/meldonization}{ifanr/AppSolution Genius Writer}
\item[2016] \href{http://www.diversitycenter.org/}{Volunteer at Diversity Center Santa Cruz County}
\item[2017] \href{https://github.com/EXONJU/ExoPlanetList}{Exoplanets Database} 与 \href{https://telegram.me/exoplanets_bot}{Telegram Query Bot} 创造者
\end{cvlist} 
\subsection*{论文发表} 
\begin{itemize}
\item[{[1]}] \textbf{Meng, Zeyang}, Zhou, Xu, Zhang, Hui, Zhou, Jilin, Wang, Songhu, Ma, Jun, Zhang, Tianmeng, Fan, Zhou, and Zou, Hu:  \href{http://adsabs.harvard.edu/abs/2013PASP..125.1015M}{\textit{Ghost Image Correction in CSTAR Photometry}}, \pasp, 125, 1015 (2013) 
\item[{[2]}] \textbf{Meng, Zeyang}, Quillen, Alice C., Bell, Cameron P.~M., Mamajek, Eric E., Scott, Erin L., and Zhou, Ji-Lin:  \href{http://adsabs.harvard.edu/abs/2014MNRAS.441.3733M}{\textit{A search for eclipsing binaries that host discs}}, \mnras, 441, 3733 (2014)  
\item[{[3]}] \textbf{Meng, Zeyang}, Xie, Ji-Wei, and Zhou, Ji-Lin:  \href{http://adsabs.harvard.edu/abs/2014IAUS..293..174M}{\textit{Planetary Survival and Ejection in Transient Multiple Star Systems}}, Formation, Detection, and Characterization of Extrasolar Habitable Planets, 293, 174 (2014) 
\item[{[4]}] \textbf{Meng, Zeyang}, D.N.C. Lin, T.M. Rogers and Zhou, Ji-Lin:  \textit{Evidence of tides as irrelevant roles in stellar obliquity damping by hot Jupiters}, in prep.  (2017)
\end{itemize}
此外共有非一作 SCI 文章 6 篇,其中包括变源搜索与掩食行星候选体搜索(罗列如下)。更详细的列表请参见\href{https://www.researchgate.net/profile/Zeyang_Meng/publications}{此链接}。
\begin{itemize}
\item[{[5]}] Yang, Ming, Zhang, Hui, Wang, Songhu, and 26 co-authors including Meng, Zeyang:  \href{http://adsabs.harvard.edu/abs/2015ApJS..217...28Y}{\textit{Eclipsing Binaries From the CSTAR Project at Dome A, Antarctica}}, ApJS, 217, 28 (2015) 
\item[{[6]}]  Quillen, Alice C., Ciocca, Marco, Carlin, Jeffrey L., Bell, Cameron P.~M., and Meng, Zeyang:  \href{http://adsabs.harvard.edu/abs/2014MNRAS.441.2691Q}{\textit{Variability in the 2MASS calibration fields: a search for transient obscuration events}}, MNRAS, 441, 2691 (2014) 
\item[{[7]}]  Wang, Songhu, Zhang, Hui, Zhou, Ji-Lin, and 24 co-authors including Meng, Zeyang:  \href{http://adsabs.harvard.edu/abs/2014ApJS..211...26W}{\textit{Planetary Transit Candidates in the CSTAR Field: Analysis of the 2008 Data}}, ApJS, 211, 26 (2014) 
\end{itemize}
\vfill
\cvplace{\hfill 南京,丁酉年二月十四}
\end{cv}

\end{resume}