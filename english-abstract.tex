% 设置论文的英文摘要
% 设置英文摘要页面的论文标题及副标题的第一行。
% 此属性可选,其默认值为使用|\englishtitle|命令所设置的论文标题
%% \englishabstracttitlea{}
% 设置英文摘要页面的论文标题及副标题的第二行。
% 此属性可选,其默认值为空白
%% \englishabstracttitleb{}
% 论文的英文摘要


\begin{englishabstract}

As one of the most active and fast growing field in astronomy, exoplanetology has boosted itself in the 
past few decades through stable observational facilities, large space coverage and long time baseline. 
Ever since the first discovery of the first hot Jupiter (51 Peg b) detected around solar-type star, thousands 
of newly discovered exoplanets has been detected nearby our solar system. It is the detailed observations 
as well as accurate photometric pipelines that has been made possible for us to reveal the nature of both our 
solar system and alien worlds.

We began by analyzing ghost image within a total of over 300,000 images observed during the Polar Night of 
2008 at Dome A site (Antarctica). The Chinese Small Telescope ARray (CSTAR) gathered $i$-band photometric 
information of the 20 deg$^2$ sky vicinity around the South Pole automatically and continuously. We carry out 
a series of elaborate analyses and study on the origin and influence of the ghost images in each frame. The 
point source catalog has also been amended by removing the ghost image effects from the real overlapped 
stars. Thus, we are able to provide a generalized ghost reduction pipeline and improve the photometric 
precision of the stars in the CSTAR FOV for astronomic researches like transiting exoplanets, variable stars 
and stellar flares. Importance has been drawn from our work for the next generation surveys in Dome A site, 
like Antarctic Survey Telescopes (AST3-2).  

Then in our next chapter, new search for systems hosting eclipsing discs are introduced by using a complete 
sample of eclipsing binaries (EBs) in the third phase of the Optical Gravitational Lensing Experiment (OGLE-III). 
Within a subsample of 2823 high-cadence, high-photometric 
precision and large eclipsing depth detached EBs previously identified in the Large Magellanic Cloud (LMC), 
we find that the skewness and kurtosis of the light-curve magnitude distribution within the primary eclipse 
can distinguish EBs with a complex-shaped eclipse from those without. Two systems with previously identified 
eclipsing discs (OGLE-LMC-ECL-11893 and OGLE-LMC-ECL-17782) are identified with near zero skewness 
($|S| < 0.5$) and positive kurtosis. No additional eclipsing disc systems were found in the OGLE-III LMC, 
Small Magellanic Cloud or Galactic Disc EB light curves. We estimate that the fraction of detached early-type 
LMC EBs (which have a primary with an I-band magnitude brighter than $\simeq$19 mag) that exhibit 
atypical eclipses and so could host a disc is approximately 1/1000. As circumstellar disc lifetimes are short, 
we expected to primarily find eclipsing discs around young stars. In addition, as there is more room for a disc 
in a widely separated binary and because a disc close to a luminous star would be above the dust sublimation 
temperature, we expected to primarily find eclipsing discs in long-period binaries. However, 
OGLE-LMC-ECL-17782 is a 13.3 d period B star system with a transient and hot ($\sim$6000 K, $\sim$0.1 au 
radius) disc and Scott et al. (2014) estimate an age of 150 Myr for OGLE-LMC-ECL-11893. Both discs are 
unexpected in the EB sample and impel explanation.

In chapter 4, known information of exoplanets are gathered for statistical ways of testing hot Jupiter theories.
Recent observations of hot Jupiter obliquities has shown a dichotomy between those planets around hot and 
cool stars: hot Jupiters around hot stars have random obliquities, while those around cool stars are generally
aligned.  This dichotomy has been used as evidence that hot Jupiters form by a scattering process, with 
variations between hot and cool stars due to variations in the tidal dissipation efficiency (Q$_\tif{s}\prime$-value)
and/or the stellar rotation speed between those stars.  In particular, it is expected that cool stars have more 
efficient tidal dissipation and slower rotation and therefore, can re-align more efficiently.  We revisit these 
scenarios taking into account the effects of equilibrium tides and stellar winds.  We find that obliquity is anti-
correlated with stellar spin around hot stars and that there is no correlation between stellar spin and orbital 
period for systems around cool stars.  Both of these findings are inconsistent with the hypothesis that slowly 
spinning stars may re-align their obliquity more easily.  Our analyses demonstrates that a single value of Q$_
\tif{s}\prime$ can not explain all hot or cool star systems.  Finally, using planets locked outside co-rotation with low 
obliquity we are able to constrain Q$_\tif{s}^\prime$ to be greater than 10$^{6}$. 

Last, Conclusion remarks are drawn in our final chapter by summarizing current and future plans of detecting 
more exoplanets and characterizing each individual planet. More physical processes and properties are in great 
demands. By placing samples of exoplanets within the historical environment of planetary formation, it could be 
possible to unlock more puzzles of our solar system and life itself in the near future.
 

% 英文关键词。关键词之间用英文半角逗号隔开,末尾无符号。
\englishkeywords{methods: data analysis, binaries: eclipsing, catalogs, planets and satellites: dynamical evolution and stability, planet–star interactions }
\end{englishabstract}


